\documentclass[12pt]{article}

%***************************************************************************************************
% Math
\usepackage{fancyhdr} 
\usepackage{amsfonts}
\usepackage{amsmath}
\usepackage{amssymb}
\usepackage{amsthm}
\usepackage{dsfont}

%***************************************************************************************************
% Macros
\usepackage{calc}

%***************************************************************************************************
% Commands and Custom Variables	
\newcommand{\problem}[1]{\hspace{-4 ex} \large \textbf{Problem #1} }
\let\oldemptyset\emptyset
\let\emptyset\varnothing
\newcommand{\norm}[1]{\left\lVert#1\right\rVert}
\newcommand{\sint}{\text{s}\kern-5pt\int}
\newcommand{\powerset}{\mathcal{P}}
\renewenvironment{proof}{\hspace{-4 ex} \emph{Proof}:}{\qed}
\newcommand{\RR}{\mathbb{R}}
\newcommand{\NN}{\mathbb{N}}
\newcommand{\QQ}{\mathbb{Q}}
\newcommand{\ZZ}{\mathbb{Z}}
\newcommand{\CC}{\mathbb{C}}


%***************************************************************************************************
%page
\usepackage[margin=1in]{geometry}
\usepackage{setspace}
%\doublespacing
\allowdisplaybreaks
\pagestyle{fancy}
\fancyhf{}
\rhead{Shaw \space \thepage}
\setlength\parindent{0pt}

%***************************************************************************************************
%Code
\usepackage{listings}
\usepackage{courier}
\lstset{
	language=Python,
	showstringspaces=false,
	formfeed=newpage,
	tabsize=4,
	commentstyle=\itshape,
	basicstyle=\ttfamily,
}

%***************************************************************************************************
%Images
\usepackage{graphicx}
\graphicspath{ {images/} }
\usepackage{float}

%tikz
\usepackage[utf8]{inputenc}
\usepackage{pgfplots}
\usepgfplotslibrary{groupplots}

%***************************************************************************************************
%Hyperlinks
%\usepackage{hyperref}
%\hypersetup{
%	colorlinks=true,
%	linkcolor=blue,
%	filecolor=magenta,      
%	urlcolor=cyan,
%}

\begin{document}
	\thispagestyle{empty}
	
	\begin{flushright}
		Sage Shaw \\
		m527 - Fall 2017 \\
		\today
	\end{flushright}
	
{\large \textbf{HW 7}}\bigbreak

\problem{1)} Let $D$ be the sphere of radius $a$ centered at the origin. Let $F$ be the velocity field $\textbf{F} = [x \ y \ z]^T$. Compute the flux across the boundary of $D$ in two substatially different ways. \bigbreak

	\emph{Directly:} Note that the normal vector at every point on the surface is $n = \tfrac{1}{a}[x \ y \ z]^T$. Let $S$ be the boundary of $D$. Then
	\begin{align*}
		\text{flux} & = \iint_S F \cdot n d\sigma \\
		& = \iint_S \tfrac{1}{a}(x^2 + y^2 + z^2) d\sigma \\
		& = \iint_S \tfrac{1}{a}a^2 d\sigma \\
		& = a \iint_S d\sigma \\
		& = 4 \pi a^3
	\end{align*}
	
	\bigbreak
	
	\emph{Divergence Theorem:} Note that $\text{div}(F) = 1+1+1 = 3$. Then
	\begin{align*}
	\text{flux} & = \iiint_D \text{div}(F) dV \\
	& = \iiint_D 3 dV \\
	& = 3 \iiint_D 1 dV \\
	& = 3 \tfrac{4}{3}\pi a^3 \\
	& = 4 \pi a^3 
	\end{align*}
	
\problem{2)} Let $S$ be the surface that is the top half of the sphere of radus $a$, centered at the origin. Let $F$ be the velocity field $F = [y, -x, 0]^T$. Compute the counterclockwise (when viewed from above) circulation of $F$ around the boundary curve of $S$ in two substantially different ways. \bigbreak

	Note that this vector field and the boundary of $S$ are planar, thus we need only use the 2-dimentional formulae for our calculations.
	\emph{Directly:} Let $r(t) = [a\cos(t) \ a\sin(t)]^T$ for $-\pi \leq t \leq \pi$. Then 
	\begin{align*}
		\text{circ} &= \int_C M dx + N dy \\
		& = \int_{-\pi}^\pi a\sin(t) (-a\sin(t)) + (-a\cos(t))(a\cos(t)) dt \\
		& = \int_{-\pi}^\pi a^2(-sin^2(t)-cos^2(t)) dt \\
		& = \int_{-\pi}^\pi -a^2 dt \\
		& = -2\pi a^2
	\end{align*}
	
	\emph{Green's Theorem:} The boundary of $S$ is the same as the bounary of $D$ where $D$ is the disk in the $xy$-plane with radius $a$ and centered at the origin. Thus we can calculate the circulation as follows
	\begin{align*}
		\text{circ} & = \iint_D \frac{\partial N}{\partial x} - \frac{\partial M}{\partial y} dA \\
		& = \iint_D -1 - 1 dA \\
		& = -2 \iint_D dA \\
		& = -2 \pi a^2
	\end{align*}
	
\problem{3)} Let $f$ be harmonic througout $D \subseteq \RR^3$. Let $S$ be the boundary of $D$. Let $n$ be the outward pointing unit vector on $S$. Let $\tfrac{\partial f}{\partial n}$ be the directional derivative of $f$ in the direction of $n$, known as the normal derivative of $f$. Prove that 
$$
\iint_S \frac{\partial f}{\partial n} d \sigma = 0
$$

	\begin{proof}
		The directional derivative is defined as $\tfrac{\partial f}{\partial n} = \nabla f \cdot \tfrac{n}{\vert n \vert}$. Thus by definition 
		$\iint_S \frac{\partial f}{\partial n} d \sigma = $ flux of $\nabla f$ over $S$ $ = \iiint_D \nabla \cdot \nabla f dV$ by the divergence theorem. But $\nabla \cdot \nabla f = \tfrac{\partial^2 f}{\partial^2 x} + \tfrac{\partial^2 f}{\partial^2 y} + \tfrac{\partial^2 f}{\partial^2 z} = 0$ by definition of a harmonic function. Thus
		$$
		\iint_S \frac{\partial f}{\partial n} d \sigma = 0
		$$
	\end{proof}

\end{document}
