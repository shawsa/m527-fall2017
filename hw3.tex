\documentclass[12pt]{article}


% Math		****************************************************************************************
\usepackage{fancyhdr} 
\usepackage{amsfonts}
\usepackage{amsmath}
\usepackage{amsthm}
\usepackage{dsfont}

% Macros	****************************************************************************************
\usepackage{calc}

% Commands	****************************************************************************************
\newcommand{\problem}[1]{\hspace{-4 ex} \large \textbf{#1}\\}

%page		****************************************************************************************
\usepackage[margin=1in]{geometry}
\usepackage{setspace}
\doublespacing
\pagestyle{fancy}
\fancyhf{}
\rhead{Shaw \space \thepage}
\setlength\parindent{0pt}


%Images		****************************************************************************************
\usepackage{graphicx}
\graphicspath{ {images/} }


\begin{document}
	\thispagestyle{empty}
	
	\begin{flushright}
		Sage Shaw \\
		m527 - Fall 2017 \\
		\today
	\end{flushright}
	
{\large \textbf{HW 3: Section 5.4: \#4, 16, 17, 20; and Section 5.5: \#4 }}\bigbreak
\large{\textbf{Section 5.4}}\\
\problem{\#4} Find a general solution in terms of $J_v$ and $J_{-v}$.
\begin{align}\label{eq1}
y^{\prime\prime} + (e^{-2x}-\frac{1}{9})y = 0
\end{align}
	We would like to manipulate this so that it matches Bessel's equation
	$$
	x^2y^{\prime\prime} + xy^\prime +(x^2-v^2)y = 0
	$$
	To do so we will substitute like so: $z=e^{-x}$. Then we can derive the following:
	\begin{align*}
		\frac{dz}{dx} & = -e^{-x} = -z &  \frac{dy}{dx} & = \frac{dy}{dz}\frac{dz}{dx}= -e^{-x}\frac{dy}{dz}\\
		\frac{d^2z}{dx^2} & = e^{-x} = z & \frac{d^2y}{dx^2} & = \frac{d^2z}{dx^2}\frac{dy}{dz} + \Big(\frac{dz}{dx}\Big)^2\frac{d^2y}{dz^2}= z\frac{dy}{dz} + z^2\frac{d^2y}{dz^2}
	\end{align*}
	Substituting into (\ref{eq1}) we have
	$$
	\Big(z\frac{dy}{dz} + z^2\frac{d^2y}{dz^2}\Big) + \Big(z^2 - \Big(\frac{1}{3}\Big)^2\Big)y = 0
	$$
	Rearranging, we obtain an equation in the form of Bessel's equation:
	\begin{align}\label{eq2}
		z^2\frac{d^2y}{dz^2} + z\frac{dy}{dz} + \Big(z^2 - \Big(\frac{1}{3}\Big)^2\Big)y & = 0
	\end{align}
	We know that $J_{\frac{1}{3}}(z)$ and $J_{-\frac{1}{3}}(z)$ are solutions to (\ref{eq2}) since $\frac{1}{3} \notin \mathbb{Z}$. Substituting $z=e^{-x}$ we obtain our general solution
	$$
	y(x) = C_1J_{\frac{1}{3}}(e^{-x}) + C_2J_{-\frac{1}{3}}(e^{-x})
	$$
	In Table 1 are shown several example solutions to our differential equation.
	\begin{table}
		\centering
		\begin{tabular}{cc}
			\includegraphics[width=.5\textwidth]{hw3_figure_1} & \includegraphics[width=.5\textwidth]{hw3_figure_2} \\
			$C_1=1$ and $C_2=0$ & $C_1=0$ and $C_2=1$ \\
			\includegraphics[width=.5\textwidth]{hw3_figure_3} & \includegraphics[width=.5\textwidth]{hw3_figure_4} \\
			$C_1=1$ and $C_2=1$ & $C_1=.5$ and $C_2=.3$ \\
		\end{tabular}
		\caption{Example solutions to equation (\ref{eq1})}
		\label{tbl:table_of_figures}
	\end{table} 
	
\problem{\#16} Show that $y-uv$ with $v(x)=e^{-\frac{1}{2}\int p(x)dx}$ and the ODE \\$y^{\prime\prime} + p(x)y^\prime + q(x)y=0$ gives the ODE $$u^{\prime\prime} + \Big[q(x) - \frac{1}{4}p(x)^2 - \frac{1}{2}p^\prime(x) \Big]u = 0$$

	Note that $v^\prime = e^{-\frac{1}{2}\int p dx} (-\frac{1}{2})p = -\frac{1}{2}pv$. Then we can derive the following
	\begin{align*}
		y^\prime & = u^\prime v + -\frac{1}{2}pvu \\
		y^{\prime\prime} & = u^{\prime\prime} v - \frac{1}{2}pvu^\prime -\frac{1}{2}p^\prime uv -\frac{1}{2}p u^\prime v - \frac{1}{4}p^2 u v\\
	\end{align*}
	Substituting into our ODE we have
	\begin{align*}
		u^{\prime\prime} v - \frac{1}{2}pvu^\prime -\frac{1}{2}p^\prime uv -\frac{1}{2}p u^\prime v + \frac{1}{4}p^2 u v + p(u^\prime v + -\frac{1}{2}pvu) + quv & = 0 \\
		v \Big(u^{\prime\prime} - \frac{1}{2}pu^\prime -\frac{1}{2}p^\prime u -\frac{1}{2}p u^\prime + \frac{1}{4}p^2 u + p(u^\prime + -\frac{1}{2}pu) + qu \Big) & = 0 \\
		v \Big(u^{\prime\prime} -\frac{1}{2}p^\prime u - p u^\prime + \frac{1}{4}p^2 u + pu^\prime + -\frac{1}{2}p^2u) + qu \Big) & = 0 \\
		v \Big(u^{\prime\prime} -\frac{1}{2}p^\prime u - \frac{1}{4}p^2 u + qu \Big) & = 0 \\
		v \Big(u^{\prime\prime} + \Big[ q  - \frac{1}{4}p^2 - \frac{1}{2}p^\prime \Big]u \Big) & = 0
	\end{align*}
	Since $v(x)=e^{-\frac{1}{2}\int p(x)dx} \neq 0$ we arrive at our desired ODE
	$$
	u^{\prime\prime} + \Big[ q  - \frac{1}{4}p^2 - \frac{1}{2}p^\prime \Big]u = 0
	$$
	
\problem{\#17} Show that for Bessel's equation, the substitution in problem 16 is $y=ux^{-1/2}$ and gives $$x^2u^{\prime\prime} + (x^2 + \frac{1}{4} - v^2)u=0$$

	Consider Bessel's equation $x^2y^{\prime\prime} + xy^\prime + (x^2-v^2)y = 0$. Dividing by $x^2$ we can put this in the form of the ODE in problem 16
	$$
	y^{\prime\prime} + x^{-1}y^\prime + \Big(1-\frac{v^2}{x^2} \Big)y = 0
	$$
	Let $q(x) = 1-\frac{v^2}{x^2}$ and $p(x) = x^{-1}$. Then let
	\begin{align*}
		\xi(x) = e^{-\frac{1}{2}\int p(x)dx} & = e^{-\frac{1}{2}\int x^{-1}dx} \\
		& = e^{-\frac{1}{2}ln(x)} \\
		& = e^{ln(x^{-\frac{1}{2}})} \\
		& = x^{-\frac{1}{2}}
	\end{align*}
	Note also that
	\begin{align*}
		p(x)^2 & = x^{-2} \\
		p^\prime(x) & = -x^{-2}
	\end{align*}
	Letting $y = u\xi$ we can substitute into our ODE as in problem 16 to obtain
	\begin{align*}
		u^{\prime\prime} + \Big[ q  - \frac{1}{4}p^2 - \frac{1}{2}p^\prime \Big]u  & = 0 \\
		u^{\prime\prime} + \Big[ \Big(1-\frac{v^2}{x^2} \Big)  - \frac{1}{4}x^{-2} + \frac{1}{2}x^{-2} \Big]u  & = 0 \\
		u^{\prime\prime} + \Big[ \Big(1-\frac{v^2}{x^2} \Big)  + \frac{1}{4}x^{-2} \Big]u  & = 0 \\
		x^2u^{\prime\prime} + \Big(x^2-v^2 + \frac{1}{4} \Big)u  & = 0 \\
	\end{align*}

\end{document}
