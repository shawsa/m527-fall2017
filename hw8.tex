\documentclass[12pt]{article}

%***************************************************************************************************
% Math
\usepackage{fancyhdr} 
\usepackage{amsfonts}
\usepackage{amsmath}
\usepackage{amssymb}
\usepackage{amsthm}
%\usepackage{dsfont}

%***************************************************************************************************
% Macros
\usepackage{calc}

%***************************************************************************************************
% Commands and Custom Variables	
\newcommand{\problem}[1]{\hspace{-4 ex} \large \textbf{Problem #1} }
\let\oldemptyset\emptyset
\let\emptyset\varnothing
\newcommand{\norm}[1]{\left\lVert#1\right\rVert}
\newcommand{\sint}{\text{s}\kern-5pt\int}
\newcommand{\powerset}{\mathcal{P}}
\renewenvironment{proof}{\hspace{-4 ex} \emph{Proof}:}{\qed}
\newcommand{\RR}{\mathbb{R}}
\newcommand{\NN}{\mathbb{N}}
\newcommand{\QQ}{\mathbb{Q}}
\newcommand{\ZZ}{\mathbb{Z}}
\newcommand{\CC}{\mathbb{C}}


%***************************************************************************************************
%page
\usepackage[margin=1in]{geometry}
\usepackage{setspace}
%\doublespacing
\allowdisplaybreaks
\pagestyle{fancy}
\fancyhf{}
\rhead{Shaw \space \thepage}
\setlength\parindent{0pt}

%***************************************************************************************************
%Code
\usepackage{listings}
\usepackage{courier}
\lstset{
	language=Python,
	showstringspaces=false,
	formfeed=newpage,
	tabsize=4,
	commentstyle=\itshape,
	basicstyle=\ttfamily,
}

%***************************************************************************************************
%Images
\usepackage{graphicx}
\graphicspath{ {images/} }
\usepackage{float}

%tikz
\usepackage[utf8]{inputenc}
\usepackage{pgfplots}
\usepgfplotslibrary{groupplots}

%***************************************************************************************************
%Hyperlinks
%\usepackage{hyperref}
%\hypersetup{
%	colorlinks=true,
%	linkcolor=blue,
%	filecolor=magenta,      
%	urlcolor=cyan,
%}

\begin{document}
	\thispagestyle{empty}
	
	\begin{flushright}
		Sage Shaw \\
		m527 - Fall 2017 \\
		\today
	\end{flushright}
	
{\large \textbf{HW 8}}\bigbreak

%%%%%%%%%%%%%%%%%%%%%%%%%%%%%%%%%%%%%%%%%%%%%%%%%%%%%%%%%%%%%%%%%%%%%%%%%%%%%%%%%%%%%%%%%%%%%%%%%%%%%
\problem{15 from the book}

	Given the differential equation
	$$
	\frac{\partial^2u}{\partial t^2} = -c^2 \frac{\partial^4u}{\partial x^4}
	$$
	We assume that a solution is of the form $u(x,t) = F(x)G(t)$, then we have $F(x)G^{\prime\prime}(t) = -c^2 F^{(4)}(x)G(t)$ and
	$$
	\frac{G^{\prime\prime}(t)}{-c^2 G(t)} = \frac{F^{(4)}(x)}{F(x)} = \beta^4
	$$
	for some constant $\beta$. This is because each side of the equation is equal for every value of $x$ and every value of $t$. Then we have that
	\begin{align}
		F^{(4)}- \beta^4F & = 0 \label{eq1} \\
		G^{\prime\prime}+ c^2\beta^4G & = 0 \label{eq2}
	\end{align}
	both ordinary differential equations. Suppose that a solution of (\ref{eq1}) is of the form $F(x) = e^{rx}$. Then substituting into (\ref{eq1}) we get $r^4e^{rx} - \beta^4e^{rx} = 0$, and $(r^4-\beta^4)e^{rx}=0$. Since $e^{rx} \neq 0$ we have that $r^4-\beta^4=0$ and that $r = \pm \beta, \pm i\beta$. These roots of the indical equation tell us that $\sinh(\beta x), \cosh(\beta x), \sin(\beta x),$ and $\cos(\beta x)$ are all solutions to (\ref{eq1}). Since they are linearly independent we know that the general solution of (\ref{eq1}) is
	\begin{align}
		F(x) = A \cos(\beta x) + B \sin(\beta x) + C \cosh(\beta x) + D \sinh(\beta x) \label{eq3}
	\end{align}
	These conditions however place no restrictions on $\beta$. Intuitively we expect that if a beam experiences bending, or displacement, that the natrual reaction of the beam will be to return to it's unbent state. We would expect then that $\beta^4$ in (\ref{eq2}) is a positive value (i.e. $G$ and $G^{\prime\prime}$ are in opposite directions). Thus
	\begin{align}
		G(t) = a\cos(c\beta^2 t) + b\sin(c\beta^2 t) \label{eq4}
	\end{align}
	is a general solution to (\ref{eq2}).
	
%%%%%%%%%%%%%%%%%%%%%%%%%%%%%%%%%%%%%%%%%%%%%%%%%%%%%%%%%%%%%%%%%%%%%%%%%%%%%%%%%%%%%%%%%%%%%%%%%%%%%
\problem{16 from the book}

	We are given the following boundary and initial conditions
	\begin{align}
		u(0,t) & = 0 \label{bc1} \\
		u(L,t) & = 0 \label{bc2} \\
		u_{xx}(0,t) & = 0 \label{bc3} \\
		u_{xx}(L,t) & = 0 \label{bc4} \\
		u_t(x,0) & = 0 \label{ic1}
	\end{align}
	Substituting (\ref{bc1}) into (\ref{eq3}) we have $0 = u(0,t) = F(0)G(t)$. Assuming that $G(t) \neq 0$ at all values of $t$, we have
	\begin{align*}
		0 & = F(0) \\
		0 & = A \cos(\beta 0) + B \sin(\beta 0) + C \cosh(\beta 0) + D \sinh(\beta 0) \\
		0 & = A + C
	\end{align*}
	We then have that $F$ has the form
	\begin{align}
		F(x) = A \cos(\beta x) + B \sin(\beta x) - A \cosh(\beta x) + D \sinh(\beta x) \label{eq10}
	\end{align}
	Substituting (\ref{bc3}) into (\ref{eq10}) we have $0 = u_{xx}(0,t) = F^{\prime\prime}(0)G(t)$. Assuming that $G(t) \neq 0$ at all values of $t$, we have
	\begin{align*}
		0 & = F^{\prime\prime}(0) \\
		0 & = -A \beta^2\cos(\beta 0) - B \beta^2 \sin(\beta 0) -A \beta^2 \cosh(\beta 0) + D \beta^2 \sinh(\beta 0) \\
		0 & = -A \beta^2 -A \beta^2 \\
		0 & = A
	\end{align*}
	We then have that $F$ has the form
	\begin{align}
		F(x) = B \sin(\beta x) + D \sinh(\beta x) \label{eq11}
	\end{align} 
	
	Substituting (\ref{bc2}) into (\ref{eq11}) we have $0 = u(L,t) = F(L)G(t)$. Assuming that $G(t) \neq 0$ at all values of $t$, we have
	\begin{align}
		0 & = F(L) \nonumber \\
		0 & = B \sin(\beta L) + D \sinh(\beta L) \label{eq12}
	\end{align}
	Similarly substituting (\ref{bc4}) into (\ref{eq11}) we have $0 = u_{xx}(L,t) = F^{\prime\prime}(L)G(t)$. Assuming that $G(t) \neq 0$ at all values of $t$, we have
	\begin{align}
	0 & = F^{\prime\prime}(L) \nonumber \\
	0 & = -B \beta^2\sin(\beta L) + D \beta^2\sinh(\beta L) \nonumber \\
	0 & = -B \sin(\beta L) + D \sinh(\beta L) \label{eq13}
	\end{align}
	Adding (\ref{eq12}) and (\ref{eq13}) we get that $0=2D \sinh (\beta L)$ and since $\beta L \neq 0$ we know that $D=0$. Lastly we subtract (\ref{eq12}) and (\ref{eq13}) (and substitute $D=0$) to get $ 0 = 2B \sin(\beta L)$. If $B=0$ we would have the trivial solution, so we will assume that $B \neq 0$. Then we have that $0 = \sin(\beta L)$ and thus $\beta = \tfrac{n\pi}{L}$ for $n \in \ZZ$. Note that since $\beta L \neq 0$. Also since $\sin(\tfrac{-n\pi}{L}) = -1\sin(\tfrac{n\pi}{L})$ we know that each $\sin(\tfrac{n\pi}{L})$ for $n \in \ZZ^-$ is linearly dependent with $\sin(\tfrac{n\pi}{L})$ for $n \in \ZZ^+$. \bigbreak
	
	Thus the set of functions
	\begin{align}
		F_n(x) = \sin(\tfrac{n\pi}{L}x) \label{eq14}
	\end{align}
	forms a basis for solutions of (\ref{eq1}) that satisfy the BC (\ref{bc1}) through (\ref{bc4}). \bigbreak
	
	Applying the initial condition (\ref{ic1}) to (\ref{eq4}) we get $u_t(x,0) = F(x)G^\prime(0)$ and assuming that $F(x) \neq 0$ for all values of $x$, we get that 
	\begin{align*}
		0 & = G^\prime(0) \\
		& = -a c \beta^2 \sin(c \beta^2 0) + b c \beta^2 \cos(c \beta^2 0) \\
		& = b c \beta^2 \\
		& = b
	\end{align*}
	Thus $G$ is of the form
	\begin{align}
		G(t) = a \cos(c \beta^2 t)
	\end{align}
	
	Substituting our criteria on $\beta$ from before we have that
	$$
		u_n(x,t) = \sin(\tfrac{n\pi}{L}x) \cos(\tfrac{cn^2 \pi^2}{L^2}t)
	$$
	Form a basis for solutions to our PDE with the given boundary conditions and initial condition.
	
%%%%%%%%%%%%%%%%%%%%%%%%%%%%%%%%%%%%%%%%%%%%%%%%%%%%%%%%%%%%%%%%%%%%%%%%%%%%%%%%%%%%%%%%%%%%%%%%%%%%%
\problem{20 from the book}

	Let $F(x)$ be of the form in (\ref{eq3}). Additionally suppose the following boundary conditions
	\begin{align}
		u(0,t) &= 0 & \implies & & F(0) &= 0 \label{bc16}\\
		u_x(0,t) &= 0 & \implies & & F^\prime(0) &= 0 \label{bc17}\\
		u_{xx}(0,t) &= 0 & \implies & & F^{\prime\prime}(0) &= 0 \label{bc18}\\
		u_{xxx}(0,t) &= 0 & \implies & & F^{\prime\prime\prime}(0) &= 0 \label{bc19}
	\end{align}
	The conditions on the right follow since we assume that $G(t) \neq 0$.
	
	Then from (\ref{eq3}) and (\ref{bc16}) we have
	\begin{align*}
		0 &= F(0) \\
		& = A \cos(\beta 0) + B \sin(\beta 0) + C \cosh(\beta 0) + D \sinh(\beta 0) \\
		& = A + C \\
		-A & = C
	\end{align*}
	Then from (\ref{eq3}) and (\ref{bc17}) we have
	\begin{align*}
		0 &= F^\prime(0) \\
		& = -A \beta \sin(\beta 0) + B \beta \cos(\beta 0) + C \beta \sinh(\beta 0) + D \beta \cosh(\beta 0) \\
		& = B \beta + D \beta \\
		-B & = D
	\end{align*}
	Now we know that $F$ is in the form
	\begin{align}
		F(x) = A \cos(\beta x) + B \sin(\beta x) - A \cosh(\beta x) - B \sinh(\beta x) \label{eq20}
	\end{align}
	From (\ref{eq20}) and (\ref{bc18}) we have
	\begin{align}
		0 &= F^{\prime\prime}(L) \nonumber \\
		& = -A \beta^2 \cos(\beta L) - B \beta^2 \sin(\beta L) - A \beta^2 \cosh(\beta L) - B \beta^2 \sinh(\beta L) \nonumber \\
		& = -\beta^2 \big[  (\cos(\beta L) + \cosh(\beta L))A + (\sin(\beta L) + \sinh(\beta L))B  \big] \nonumber \\
		0 & = (\cos(\beta L) + \cosh(\beta L))A + (\sinh(\beta L) + \sin(\beta L))B \label{eq21}
	\end{align}
	And from (\ref{eq20}) and (\ref{bc19}) we have
	\begin{align}
		0 &= F^{\prime\prime\prime}(L) \nonumber \\
		& = A \beta^3 \sin(\beta L) - B \beta^3 \cos(\beta L) - A \beta^3 \sinh(\beta L) - B \beta^3 \cosh(\beta L) \nonumber \\
		& = -\beta^3 \big[  ( -\sin(\beta L) + \sinh(\beta L))A + (\cos(\beta L) + \cosh(\beta L))B  \big] \nonumber \\
		0 & = (\sinh(\beta L) - \sin(\beta L))A + (\cos(\beta L) + \cosh(\beta L))B \label{eq22}
	\end{align}
	Equations (\ref{eq21}) and (\ref{eq22}) are two linear equations of two variables that can be written as the following matrix equation
	\begin{align}
		\begin{bmatrix}
		\cos(\beta L) + \cosh(\beta L) & \sinh(\beta L) + \sin(\beta L)\\
		\sinh(\beta L) - \sin(\beta L) & \cos(\beta L) + \cosh(\beta L)\\
		\end{bmatrix}
		\begin{bmatrix}
		A \\
		B\\
		\end{bmatrix}
		=
		\begin{bmatrix}
		0 \\
		0\\
		\end{bmatrix} \label{eq23}
	\end{align}
	We are interested in nontrivial solutions of (\ref{eq23}). That would mean that the matrix maps a nonzero vector to the zero vector. This can only happen when the determinant of the matrix is zero which leads us to our final result
	\begin{align*}
		0 & = (\cos(\beta L) + \cosh(\beta L))^2 - (\sinh(\beta L) - \sin(\beta L))(\sinh(\beta L) + \sin(\beta L)) \\
		0 & = \cos^2(\beta L) + 2\cos(\beta L)\cosh(\beta L) + \cosh^2(\beta L) - \sinh^2(\beta L) + \sin^2(\beta L) \\
		0 & = 2\cos(\beta L)\cosh(\beta L) + \cos^2(\beta L) + \sin^2(\beta L) + \cosh^2(\beta L) - \sinh^2(\beta L) \\
		0 & = 2\cos(\beta L)\cosh(\beta L) + 1 + 1 \\ 
		-1 & = \cos(\beta L)\cosh(\beta L)
	\end{align*}
	
	If we assume that $\cosh(\beta L) \cos(\beta L) = -1$ we can use any root finding method on $\cosh(\beta L) \cos(\beta L) +1 = 0$ to find an approximate value for $\beta L$. Using the secant method I calculate that $\beta L \approx 1.8751040687119611$ is accurate to within an error of $10^{-15}$. 
	
%%%%%%%%%%%%%%%%%%%%%%%%%%%%%%%%%%%%%%%%%%%%%%%%%%%%%%%%%%%%%%%%%%%%%%%%%%%%%%%%%%%%%%%%%%%%%%%%%%%%%
\problem{2}

	Given that $f(x)$ is the initial height of the guitar string as a function of $x$ the distance along the fretboard of the guitar, that passes through the points $(0,0), (0,L)$ and $(a,b)$ where $0 < a < L$ and some number $b$, we will assume that $f$ is a peicewise linear function passing through these points. Then using the point-slope formula, $f$ is given by
	
	$$
	f(x) =
	\begin{cases}
	\frac{b}{a}x & \text{ if } 0 \leq x \leq a \\
	\frac{0-b}{L-a}(x-L) & \text{ if } a \leq x \leq L \\
	\end{cases}
	$$
	
\end{document}
