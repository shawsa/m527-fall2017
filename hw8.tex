\documentclass[12pt]{article}

%***************************************************************************************************
% Math
\usepackage{fancyhdr} 
\usepackage{amsfonts}
\usepackage{amsmath}
\usepackage{amssymb}
\usepackage{amsthm}
%\usepackage{dsfont}

%***************************************************************************************************
% Macros
\usepackage{calc}

%***************************************************************************************************
% Commands and Custom Variables	
\newcommand{\problem}[1]{\hspace{-4 ex} \large \textbf{Problem #1} }
\let\oldemptyset\emptyset
\let\emptyset\varnothing
\newcommand{\norm}[1]{\left\lVert#1\right\rVert}
\newcommand{\sint}{\text{s}\kern-5pt\int}
\newcommand{\powerset}{\mathcal{P}}
\renewenvironment{proof}{\hspace{-4 ex} \emph{Proof}:}{\qed}
\newcommand{\RR}{\mathbb{R}}
\newcommand{\NN}{\mathbb{N}}
\newcommand{\QQ}{\mathbb{Q}}
\newcommand{\ZZ}{\mathbb{Z}}
\newcommand{\CC}{\mathbb{C}}


%***************************************************************************************************
%page
\usepackage[margin=1in]{geometry}
\usepackage{setspace}
%\doublespacing
\allowdisplaybreaks
\pagestyle{fancy}
\fancyhf{}
\rhead{Shaw \space \thepage}
\setlength\parindent{0pt}

%***************************************************************************************************
%Code
\usepackage{listings}
\usepackage{courier}
\lstset{
	language=Python,
	showstringspaces=false,
	formfeed=newpage,
	tabsize=4,
	commentstyle=\itshape,
	basicstyle=\ttfamily,
}

%***************************************************************************************************
%Images
\usepackage{graphicx}
\graphicspath{ {images/} }
\usepackage{float}

%tikz
\usepackage[utf8]{inputenc}
\usepackage{pgfplots}
\usepgfplotslibrary{groupplots}

%***************************************************************************************************
%Hyperlinks
%\usepackage{hyperref}
%\hypersetup{
%	colorlinks=true,
%	linkcolor=blue,
%	filecolor=magenta,      
%	urlcolor=cyan,
%}

\begin{document}
	\thispagestyle{empty}
	
	\begin{flushright}
		Sage Shaw \\
		m527 - Fall 2017 \\
		\today
	\end{flushright}
	
{\large \textbf{HW 8}}\bigbreak

%%%%%%%%%%%%%%%%%%%%%%%%%%%%%%%%%%%%%%%%%%%%%%%%%%%%%%%%%%%%%%%%%%%%%%%%%%%%%%%%%%%%%%%%%%%%%%%%%%%%%
\problem{15 from the book}

	Given the differential equation
	$$
	\frac{\partial^2u}{\partial t^2} = -c^2 \frac{\partial^4u}{\partial x^4}
	$$
	We assume that a solution is of the form $u(x,t) = F(x)G(t)$, then we have $F(x)G^{\prime\prime}(t) = -c^2 F^{(4)}(x)G(t)$ and
	$$
	\frac{G^{\prime\prime}(t)}{-c^2 G(t)} = \frac{F^{(4)}(x)}{F(x)} = \beta^4
	$$
	for some constant $\beta$. This is because each side of the equation is equal for every value of $x$ and every value of $t$. Then we have that
	\begin{align}
		F^{(4)}- \beta^4F & = 0 \label{eq1} \\
		G^{\prime\prime}+ c^2\beta^4G & = 0 \label{eq2}
	\end{align}
	both ordinary differential equations. Suppose that a solution of (\ref{eq1}) is of the form $F(x) = e^{rx}$. Then substituting into (\ref{eq1}) we get $r^4e^{rx} - \beta^4e^{rx} = 0$, and $(r^4-\beta^4)e^{rx}=0$. Since $e^{rx} \neq 0$ we have that $r^4-\beta^4=0$ and that $r = \pm \beta, \pm i\beta$. These roots of the indical equation tell us that $\sinh(\beta x), \cosh(\beta x), \sin(\beta x),$ and $\cos(\beta x)$ are all solutions to (\ref{eq1}). Since they are linearly independent we know that the general solution of (\ref{eq1}) is
	\begin{align}
		F(x) = A \cos(\beta x) + B \sin(\beta x) + C \cosh(\beta x) + D \sinh(\beta x) \label{eq3}
	\end{align}
	These conditions however place no restrictions on $\beta$. Intuitively we expect that if a beam experiences bending, or displacement, that the natrual reaction of the beam will be to return to it's unbent state. We would expect then that $\beta^4$ in (\ref{eq2}) is a positive value (i.e. $G$ and $G^{\prime\prime}$ are in opposite directions). Thus
	\begin{align}
		G(t) = a\cos(c\beta^2 t) + b\sin(c\beta^2 t) \label{eq4}
	\end{align}
	is a general solution to (\ref{eq2}).
	
%%%%%%%%%%%%%%%%%%%%%%%%%%%%%%%%%%%%%%%%%%%%%%%%%%%%%%%%%%%%%%%%%%%%%%%%%%%%%%%%%%%%%%%%%%%%%%%%%%%%%
\problem{16 from the book}

	We are given the following boundary and initial conditions
	\begin{align}
		u(0,t) & = 0 \label{bc1} \\
		u(L,t) & = 0 \label{bc2} \\
		u_{xx}(0,t) & = 0 \label{bc3} \\
		u_{xx}(L,t) & = 0 \label{bc4} \\
		u_t(x,0) & = 0 \label{ic1}
	\end{align}
	Substituting (\ref{bc1}) into (\ref{eq3}) we have $0 = u(0,t) = F(0)*G(t)$. Assuming that $G(t) \neq 0$ at all values of $t$, we have
	\begin{align*}
		0 & = F(0) \\
		0 & = A \cos(\beta 0) + B \sin(\beta 0) + C \cosh(\beta 0) + D \sinh(\beta 0) \\
		0 & = A + C
	\end{align*}
	We then have that $F$ has the form
	\begin{align}
		F(x) = A \cos(\beta x) + B \sin(\beta x) - A \cosh(\beta x) + D \sinh(\beta x) \label{eq5}
	\end{align}
	Substituting (\ref{bc3}) into (\ref{eq5}) we have $0 = u_{xx}(0,t) = F^{\prime\prime}(0)*G(t)$. Assuming that $G(t) \neq 0$ at all values of $t$, we have
	\begin{align*}
		0 & = F^{\prime\prime}(0) \\
		0 & = -A \beta^2\cos(\beta 0) - B \beta^2 \sin(\beta 0) -A \beta^2 \cosh(\beta 0) + D \beta^2 \sinh(\beta 0) \\
		0 & = -A \beta^2 -A \beta^2 \\
		0 & = A
	\end{align*}
	We then have that $F$ has the form
	\begin{align}
		F(x) = B \sin(\beta x) + D \sinh(\beta x) \label{eq6}
	\end{align} 
	
\end{document}
