\documentclass[12pt]{article}


% Math		****************************************************************************************
\usepackage{fancyhdr} 
\usepackage{amsfonts}
\usepackage{amsmath}
\usepackage{amsthm}
\usepackage{dsfont}

% Commands	****************************************************************************************
%\newcommand{\problem}[1]{\hspace{-\widthof{#1}} \textbf{#1}}
\newcommand{\problem}[1]{\hspace{-4 ex} \large \textbf{#1}\\}
\newcommand{\reverseconcat}[3]{#3#2#1}

%page		****************************************************************************************
\usepackage[margin=1in]{geometry}
\usepackage{setspace}
\doublespacing
\pagestyle{fancy}
\fancyhf{}
\rhead{Shaw \space \thepage}
\setlength\parindent{0pt}


\begin{document}
	\thispagestyle{empty}
	
	\begin{flushright}
		Sage Shaw \\
		m527 - Fall 2017 \\
		\today
	\end{flushright}
	
{\large \textbf{HW 2 - Section 5.2: \#2, 4 (for $P_2(x)$ and $P_4(x)$ only), 11, and two additional problems specified below.}}\bigbreak

\problem{Section 5.2 \#2} Show that equation (7) from the text with $n=1$ gives $y_2(x)=P_1(x)=x$ and that (6) from the text gives 
\begin{align*}
y_1(x) & = 1 - x^2 - \frac{1}{3}x^4 - \frac{1}{5}x^6 - ... \\
& = 1 - \frac{1}{2}x \text{ln}\frac{1+x}{1-x} \\
\end{align*}
\\

	Equation (7) from the text is stated as follows:
	$$
	y_2(x) = x - \frac{(n-1)(n+2)}{3!}x^3 + \frac{(n-3)(n-1)(n+2)(n+4)}{5!}x^5 - ...
	$$
	Notice that in each term past $x$ there is a $(n-1)$ term in the numerator. Since $n=1$ these terms become zero and we're left with $y_2(x)=x$ when $n=1$.
	
	Equation (6) from the text is stated as follows:
	$$
	y_1(x) = 1 - \frac{n(n+1)}{2!}x^2 + \frac{(n-2)n(n+1)(n+3)}{4!}x^4 - ...
	$$
	
	Substituting in $n=1$ and observing more terms we have
	\begin{align}
		y_1(x) & = 1 - \frac{1(1+1)}{2!}x^2 + \frac{(1-2)1(1+1)(1+3)}{4!}x^4 + \nonumber \\
		& - \frac{(1-4)(1-2)1(1+1)(1+3)(1+5)}{6!}x^4 +... \nonumber \\
		& = 1 - \frac{2}{2}x^2 + \frac{(-1)1(2)(4)}{4*3*2}x^4 - \frac{(-3)(-1)1(2)(4)(6)}{6*5*4*3*2}x^4 +... \nonumber \\
		y_1(x) & = 1 - x^2 - \frac{1}{3}x^4 - \frac{1}{5}x^4 -... \label{p1_e1}
	\end{align}
	
	Recall these two Taylor Series expansions of the natural logarithm. \\
	\begin{align*}
		\text{ln}(1+x) & = \sum\limits_{m=1}^\infty (-1)^{m+1} \frac{x^m}{m}\\
		\text{ln}(1-x) & = -\sum\limits_{m=1}^\infty \frac{x^m}{m}\\
	\end{align*}
	Notice that
	\begin{align*}
		\text{ln}\frac{1+x}{1-x} & = \text{ln}(1+x) - \text{ln}(1-x) \\
		& = \sum\limits_{m=1}^\infty (-1)^{m+1} \frac{x^m}{m} + \sum\limits_{m=1}^\infty \frac{x^m}{m} \\
		& = \sum\limits_{m=1}^\infty \big( (-1)^{m+1} + 1 \big) \frac{x^m}{m} \\
		& = x + 0 + \frac{1}{3}x^3 + 0 + \frac{1}{5}x^5 + ... \\
		& = 2x + \frac{2}{3}x^3 + \frac{2}{5}x^5 + ... \\
		1 - \frac{1}{2}x\text{ln}\frac{1+x}{1-x}& = 1 -x^2 - \frac{1}{3}x^4 - \frac{1}{5}x^4 + ...
	\end{align*}
	Then from \ref{p1_e1} we have $y_2(x) = 1 - \frac{1}{2}x\text{ln}\frac{1+x}{1-x}$. \\
	
\problem{Section 5.2 \#4} Verify that the polynomials (do $P_2$ and $P_4$ only) in equation ($11^\prime$) from the text satisfy equation (1) from the text. \\

	From ($11^\prime$) in the text, we have
	\begin{align*}
		P_2(x) & = \frac{1}{2}(3x^2-1) \\
		P_4(x) & = \frac{1}{8}(35x^4-30x^2+3) 
	\end{align*}
	and we need to verify that they are solutions to $(1-x^2)y^{\prime\prime} - 2xy^\prime + n(n+1)y=0$. 
	
	Let's first consider $n=2$. Then 
	\begin{align*}
		y & = P_2(x) = \frac{1}{2}(3x^2-1) \\
		y^\prime & = 3x \\
		y^{\prime\prime} & = 3 \\
		(1-x^2)y^{\prime\prime} - 2xy^\prime + n(n+1)y & = (1-x^2)3 - 2x(3x) + 2(3)\frac{1}{2}(3x^2-1) \\
		& = 3-3x^2 - 6x^2 + 9x^2-3 \\
		& = 0
	\end{align*}
	Now let's consider $n=4$. Then 
	\begin{align*}
	y & = P_4(x) = \frac{1}{8}(35x^4-30x^2+3) \\
	y^\prime & = \frac{1}{8}(140x^3-60x) \\
	y^{\prime\prime} & = \frac{1}{8}(420x^2-60) \\
	(1-x^2)y^{\prime\prime} &- 2xy^\prime + n(n+1)y \\ 
	& = (1-x^2)\frac{1}{8}(420x^2-60) - 2x\frac{1}{8}(140x^3-60x) + 4(5)\frac{1}{8}(35x^4-30x^2+3) \\
	& = \frac{1}{8} \big[  -420x^4 + 480x^2 - 60 - 280x^4 + 120x^2 +700x^4 - 600x^2 + 60 \big] \\
	& = 0
	\end{align*}
	
\problem{Section 5.2 \#11} 

\end{document}
