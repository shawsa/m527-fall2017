\documentclass[12pt]{article}

%%%%%%%%%%%%%%%%%%%%%%%%%%%%%%%%%%%%%%%%%%%%%%%%%%%%%%%%%%%%%%%%%%%%%%%%%%%%%%%%%%%%%%%%%%%%%%%%%%%%
% Math
\usepackage{fancyhdr} 
\usepackage{amsfonts}
\usepackage{amsmath}
\usepackage{amssymb}
\usepackage{amsthm}
%\usepackage{dsfont}

%%%%%%%%%%%%%%%%%%%%%%%%%%%%%%%%%%%%%%%%%%%%%%%%%%%%%%%%%%%%%%%%%%%%%%%%%%%%%%%%%%%%%%%%%%%%%%%%%%%%
% Macros
\usepackage{calc}

%%%%%%%%%%%%%%%%%%%%%%%%%%%%%%%%%%%%%%%%%%%%%%%%%%%%%%%%%%%%%%%%%%%%%%%%%%%%%%%%%%%%%%%%%%%%%%%%%%%%
% Commands and Custom Variables	
\newcommand{\problem}[1]{\hspace{-4 ex} \large \textbf{Problem #1} }
\let\oldemptyset\emptyset
\let\emptyset\varnothing
\newcommand{\norm}[1]{\left\lVert#1\right\rVert}
\newcommand{\sint}{\text{s}\kern-5pt\int}
\newcommand{\powerset}{\mathcal{P}}
\renewenvironment{proof}{\hspace{-4 ex} \emph{Proof}:}{\qed}
\newcommand{\RR}{\mathbb{R}}
\newcommand{\NN}{\mathbb{N}}
\newcommand{\QQ}{\mathbb{Q}}
\newcommand{\ZZ}{\mathbb{Z}}
\newcommand{\CC}{\mathbb{C}}


%%%%%%%%%%%%%%%%%%%%%%%%%%%%%%%%%%%%%%%%%%%%%%%%%%%%%%%%%%%%%%%%%%%%%%%%%%%%%%%%%%%%%%%%%%%%%%%%%%%%
%page
\usepackage[margin=1in]{geometry}
\usepackage{setspace}
%\doublespacing
\allowdisplaybreaks
\pagestyle{fancy}
\fancyhf{}
\rhead{Shaw \space \thepage}
\setlength\parindent{0pt}

%%%%%%%%%%%%%%%%%%%%%%%%%%%%%%%%%%%%%%%%%%%%%%%%%%%%%%%%%%%%%%%%%%%%%%%%%%%%%%%%%%%%%%%%%%%%%%%%%%%%
%Code
\usepackage{listings}
\usepackage{courier}
\lstset{
	language=Python,
	showstringspaces=false,
	formfeed=newpage,
	tabsize=4,
	commentstyle=\itshape,
	basicstyle=\ttfamily,
}

%%%%%%%%%%%%%%%%%%%%%%%%%%%%%%%%%%%%%%%%%%%%%%%%%%%%%%%%%%%%%%%%%%%%%%%%%%%%%%%%%%%%%%%%%%%%%%%%%%%%
%Images
\usepackage{graphicx}
\graphicspath{ {images/} }
\usepackage{float}

%tikz
\usepackage[utf8]{inputenc}
\usepackage{pgfplots}
\usepgfplotslibrary{groupplots}

%%%%%%%%%%%%%%%%%%%%%%%%%%%%%%%%%%%%%%%%%%%%%%%%%%%%%%%%%%%%%%%%%%%%%%%%%%%%%%%%%%%%%%%%%%%%%%%%%%%%
%Hyperlinks
%\usepackage{hyperref}
%\hypersetup{
%	colorlinks=true,
%	linkcolor=blue,
%	filecolor=magenta,      
%	urlcolor=cyan,
%}

\begin{document}
	\thispagestyle{empty}
	
	\begin{flushright}
		Sage Shaw \\
		m527 - Fall 2017 \\
		\today
	\end{flushright}
	
{\large \textbf{HW 9}}\bigbreak

%%%%%%%%%%%%%%%%%%%%%%%%%%%%%%%%%%%%%%%%%%%%%%%%%%%%%%%%%%%%%%%%%%%%%%%%%%%%%%%%%%%%%%%%%%%%%%%%%%%%
\problem{\#2 from Sec 12.4}

	Show that the boundary conditions
	\begin{align}
		u(0,t) &= 0 & u(L,t) &= 0 \label{BC}
	\end{align}
	and the result of D'lambert's solution
	\begin{align}
		u(x,t) = \tfrac{1}{2}[f(x+ct)+f(x-ct)] \label{eq13}
	\end{align}
	requre that $f$ is odd and has period $2L$. \bigbreak
	
	\begin{proof}
		First substitute $u(0,t)=0$ into (\ref{BC}) to obtain $0 = \tfrac{1}{2}[f(ct)+f(-ct)]$. Then $f(-ct) = -f(ct)$ for any $ct$ and thus $f$ is odd. Next substitute $u(0,t)=0$ into (\ref{BC}) to obtain $0 = \tfrac{1}{2}[f(L+ct)+f(L-ct)]$. Rearanging we have
		\begin{align*}
			f(L+ct) &= -f(L-ct) \\
			f(ct+L) &= f\big(-(L-ct)\big) \text{\ \ \ \ since $f$ is odd} \\
			f(ct+L) &= f(ct-L)
		\end{align*}
		With the change of variables $ct = x+L$ we have
		$$
		f(x+2L) = f(x)
		$$
		and thus $f$ has period $2L$.
	\end{proof} \bigbreak

%%%%%%%%%%%%%%%%%%%%%%%%%%%%%%%%%%%%%%%%%%%%%%%%%%%%%%%%%%%%%%%%%%%%%%%%%%%%%%%%%%%%%%%%%%%%%%%%%%%%
\problem{\#16 from Sec 12.6: A bar with heat generation}of constant rate $H>0$ is modeled by $u_t = c^2u_{xx} + H$. Solve this problem if $L=\pi$ and the ends of the bar are kept at $0^\circ$C. \bigbreak

	First define $v(x,t)$ such that $u = v - Hx(x-\pi)/(2c^2)$. Then $v_t = u_t$ and
	\begin{align*}
		u_x & = v_x - H(2x-\pi)/(2c^2) \\
		u_xx & = v_{xx} - 2H/(2c^2)\\
		& = v_{xx} - \tfrac{H}{c^2}
	\end{align*}
	Substituting into our differential equation we obtain
	\begin{align*}
		u_t & = c^2u_{xx} + H \\
		v_t = c^2(v_{xx} - \tfrac{H}{c^2}) + H \\
		v_t = c^2v_{xx}\\
	\end{align*}
	Also $0 = u(0,t)$ implies that $v(0,t) = 0$ and $0 = u(L,t)$ implies that $v(L,t) = 0$. From earlier we know that 
	$$
	v(x,t) = \sum B_n \sin(nx) e^{-c^2n^2t}
	$$
	is the general solution to this differential equation given these boundary conditions. Thus
	$$
	u(x,t) = -\frac{H}{2c^2}x(x-\pi) + \sum B_n \sin(nx) e^{-c^2n^2t}
	$$
	is the general solution to our differential equation. 
	\bigbreak

%%%%%%%%%%%%%%%%%%%%%%%%%%%%%%%%%%%%%%%%%%%%%%%%%%%%%%%%%%%%%%%%%%%%%%%%%%%%%%%%%%%%%%%%%%%%%%%%%%%%
\problem{\#22 from Sec 12.6} Find the steady state temperature in the square plate with side lenth 24, when the left and right sides are kept at $0^\circ$C while the lower side is kept at $U_0^\circ$C and the upper side is kept at $U_1^\circ$C. \bigbreak

	First I would like to note that I'm unsure if this problem makes physical sense. If the upper side is kept at a constant tempurature $U_1^\circ$C and the left side is kept at $0^\circ$C then what is the tempurature at the upper left point of the plate? These conditions would seem to contradict eachother. \bigbreak
	
	To simplify this problem, let's consider the case when $U_0 = 0$. Then we know that a general solution to that PDE with the boundary conditions is 
	$$
	v(x,y) = \sum A_n \sin(\tfrac{n\pi}{24}x)\sinh(\tfrac{n\pi}{24}y)
	$$
	where
	\begin{align*}
		A_n & = \tfrac{2}{24\sinh(n\pi)} \int_0^{24} U_1 \sin(\tfrac{n\pi}{24}x) dx \\
		& = \tfrac{2}{24\sinh(n\pi)} U_1 \tfrac{24}{n\pi} [-\cos(\tfrac{n\pi}{24}x)]_0^{24} \\
		& = \frac{2U_1}{n\pi\sinh(n\pi)} [1 - \cos(n\pi)] \\
		& = \begin{cases}
				0 & \text{if $n$ is even} \\
				\tfrac{4U_1}{n\pi\sinh(n\pi)} & \text{if $n$ is odd}
		\end{cases}
	\end{align*}
	To simplify completely we have
	$$
	v(x,y) = \sum_{n=0}^\infty \frac{4U_1}{(2n+1)\pi\sinh((2n+1)\pi)} \sin \Big(\tfrac{(2n+1)\pi}{24}x \Big)\sinh \Big(\tfrac{(2n+1)\pi}{24}y \Big)
	$$
	or
	$$
	v(x,y) = \sum_{n=0}^\infty \frac{4}{(2n+1)\pi\sinh((2n+1)\pi)} \sin \Big(\tfrac{(2n+1)\pi}{24}x \Big) U_1 \sinh \Big(\tfrac{(2n+1)\pi}{24}y \Big)
	$$
	We next consider the case when $U_1=0$. If we look for solutions in the form of $w(x, 24-y)$ we will see that this is the same boundary value problem. Hence a solution will be of the form
	$$
	w(x,y) = \sum_{n=0}^\infty \frac{4}{(2n+1)\pi\sinh((2n+1)\pi)} \sin \Big(\tfrac{(2n+1)\pi}{24}x \Big) U_0 \sinh \Big(\tfrac{(2n+1)\pi}{24}(24-y) \Big)
	$$
	Thus a solution to our general differential equation where $U_1$ and $U_2$ may not be zero is
	$$
	u(x,y) = \sum_{n=0}^\infty A_n \sin \Big(\tfrac{(2n+1)\pi}{24}x \Big) 
	\Big[ U_1 \sinh \Big(\tfrac{(2n+1)\pi}{24}y \Big) + U_0 \sinh \Big(\tfrac{(2n+1)\pi}{24}(24-y) \Big) \Big ]
	$$
	Where $A_n = \frac{4}{(2n+1)\pi\sinh((2n+1)\pi)}$.

	
\end{document}
