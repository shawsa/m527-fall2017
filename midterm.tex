\documentclass[12pt]{article}


% Math		****************************************************************************************
\usepackage{fancyhdr} 
\usepackage{amsfonts}
\usepackage{amsmath}
\usepackage{amssymb}
\usepackage{amsthm}
%\usepackage{dsfont}

% Macros	****************************************************************************************
\usepackage{calc}

% Commands and Custom Variables	********************************************************************
\newcommand{\problem}[1]{\hspace{-4 ex} \large \textbf{Problem #1} }
\let\oldemptyset\emptyset
\let\emptyset\varnothing
\newcommand{\norm}[1]{\left\lVert#1\right\rVert}
\newcommand{\sint}{\text{s}\kern-5pt\int}
\newcommand{\powerset}{\mathcal{P}}
\renewenvironment{proof}{\hspace{-4 ex} \emph{Proof}:}{\qed}

%page		****************************************************************************************
\usepackage[margin=1in]{geometry}
\usepackage{setspace}
%\doublespacing
\allowdisplaybreaks
\pagestyle{fancy}
\fancyhf{}
\rhead{Shaw \space \thepage}
\setlength\parindent{0pt}

%Code		****************************************************************************************
\usepackage{listings}
\usepackage{courier}
\lstset{
	language=Python,
	showstringspaces=false,
	formfeed=newpage,
	tabsize=4,
	commentstyle=\itshape,
	basicstyle=\ttfamily,
}

%Images		****************************************************************************************
\usepackage{graphicx}
\graphicspath{ {images/} }

%tikz
\usepackage[utf8]{inputenc}
\usepackage{pgfplots}

%Hyperlinks	****************************************************************************************
%\usepackage{hyperref}
%\hypersetup{
%	colorlinks=true,
%	linkcolor=blue,
%	filecolor=magenta,      
%	urlcolor=cyan,
%}


\begin{document}
	\thispagestyle{empty}
	
	\begin{flushright}
		Sage Shaw \\
		m527 - Fall 2017 \\
		\today
	\end{flushright}
	
{\large \textbf{Midterm Exam}}\bigbreak

%*********************************************************************************************************************************************
\problem{1 a)} Consider the family of functions for $n = 0,1,2,3,\dots$
\begin{align} \label{p1eq1}
	f_n(x)=\cos{(n \arccos{x})}
\end{align}
Prove that every function of the form (\ref{p1eq1}) satisfies the differential equation
	
	\begin{align}\label{p1eq2}
		(1-x^2)y^{\prime\prime}-xy^\prime+n^2y=0
	\end{align}
	
	\begin{proof}
		Note that if $y = f_n(x)$, then
		\begin{align*}
			y^\prime & = -\sin{(n \arccos{x})} \tfrac{d}{dx}(n \arccos{x}) \\
			& = -\sin{(n \arccos{x})} (-n)(1-x^2)^{-\frac{1}{2}} \\
			& = n(1-x^2)^{-\frac{1}{2}}\sin{(n \arccos{x})}  \\
			y^{\prime\prime} &= -\tfrac{n}{2}(1-x^2)^{-\frac{3}{2}}(-2x)\sin{(n \arccos{x})} + n(1-x^2)^{-\frac{1}{2}}\cos{(n \arccos{x})}(-n)(1-x^2)^{-\frac{1}{2}} \\
			&= nx(1-x^2)^{-\frac{3}{2}}\sin{(n \arccos{x})} - n^2(1-x^2)^{-1}\cos{(n\arccos{x})}
		\end{align*}
		Then 
		\begin{align*}
			(1-x^2)y^{\prime\prime} & = (1-x^2) \Big( nx(1-x^2)^{-\frac{3}{2}}\sin{(n \arccos{x})} - n^2(1-x^2)^{-1}\cos{(n\arccos{x})} \Big) \\
			& = nx(1-x^2)^{-\frac{1}{2}}\sin{(n \arccos{x})} - n^2\cos{(n\arccos{x})} \\
			-xy^\prime & = -x n(1-x^2)^{-\frac{1}{2}}\sin{(n \arccos{x})} \\
			n^2y & = n^2\cos{(n \arccos{x})}
		\end{align*}
		Clearly these terms sum to zero and we have our result $(1-x^2)y^{\prime\prime}-xy^\prime+n^2y=0$. Thus $f_n$ are solutions to (\ref{p1eq2}).
		
	\end{proof}
	
%*********************************************************************************************************************************************
\problem{1 b)} Prove that the functions (\ref{p1eq1}) satisfy the following
	
	\emph{i)} $f_n(1) = 1$ \\
	\emph{ii)} $f_n(-1) = (-1)^n$ \\
	\emph{iii)} $f_{2n+1}(0) = 0$ \\
	\emph{iv)} $f_{2n}(0) = (-1)^n$ \\
	\emph{v)} $f_n(-x) = (-1)^nf_n(x)$
	
	\doublespacing
	\begin{proof}
		Clearly $f_n(1) = \cos{(n \arccos{1})} = \cos{(n*0)} = \cos{(0)} = 1$, and thus we have (\emph{i}). \\
		In a similar fashion $f_n(-1) = \cos{(n \arccos{-1})} = \cos(n\pi)$ If $n$ is even $f_n(-1) = 1$ and if $n$ is odd then $f_n(-1)=-1$. This is also the case for $(-1)^n$. Thus $f_n(-1) = (-1)^n$ and we have (\emph{ii}). \\
		Next, see that
		\begin{align*}
			f_{2n+1}(0) & = \cos \big( (2n+1) \arccos(0) \big) \\
			& = \cos \big( (2n+1) \tfrac{\pi}{2} \big) \\
			& = \cos(n\pi + \tfrac{\pi}{2}) \\
			& = 0
		\end{align*}
		and thus we have \emph{(iii)}.\\
		Next, see that
		\begin{align*}
			f_{2n}(0) & = \cos \big( (2n) \arccos(0) \big) \\
			& = \cos \big( (2n) \tfrac{\pi}{2} \big) \\
			& = \cos(n\pi)
		\end{align*}
		Thus $f_{2n}(0)=1$ if $n$ is even and $f_{2n}(0)=-1$ if $n$ is odd. This is also the case for $(-1)^n$. Therefore $f_{2n}(0) = (-1)^n$, and we have \emph{(iv)}.\\
		For the next proof, we will first need to show that $\arccos(-x) = \pi - \arccos(x)$. Let $\arccos(-x)=y$, then $-x = \cos(y)$. From the symmetry of cosine, one can see that $x = \cos(\pi-y)$. Then $\arccos(x) = \pi - y = \pi - \arccos(-x)$, and thus $\arccos(-x) = \pi - \arccos(x)$.\\
		Finally 
		\begin{align*}
			f_n(-x) & = \cos(n \arccos(-x)) \\
			& = \cos \big( n (\pi - \arccos(x)) \big) \\
			& = \cos (n\pi - n\arccos(x))
		\end{align*}
		If $n$ is even then $f_n(-x) = \cos(-n \arccos(x))$ and by the symmetry of cosine $f_n(-x) = \cos(n \arccos(x)) = (-1)^n\cos(n \arccos(x))$. If $n$ is odd, then $f_n(-x) = \cos(\pi -n \arccos(x))$ and by the symmetry of cosine $f_n(-x) = -1\cos(n\arccos(x)) = (-1)^n\cos(n \arccos(x))$. Thus we have (\emph{v}).
	\end{proof}

%*********************************************************************************************************************************************
\problem{1 c)}
	Suppose that our solution to (\ref{p1eq2}) was in the form $y = \sum \limits _{m=0}^\infty a_m x^m$. Then
	\begin{align*}
		y^\prime & = \sum \limits _{m=1}^\infty ma_{m}x^{m-1} & y^{\prime\prime} & = \sum \limits _{m=2}^\infty m(m-1)a_{m}x^{m-2} \\
	\end{align*}
	We will rearrange (\ref{p1eq2}) as follows
	\begin{align}\label{p1eq2prime}
		0 & = y^{\prime\prime} - x^2y^{\prime\prime}-xy^\prime+n^2y \tag{$2^\prime$}
	\end{align}
	We would like the summations to have the same indecies and the same powers of $x$. Consider the terms individually.
	\begin{align*}
		y^{\prime\prime} & = \sum \limits _{m=2}^\infty m(m-1)a_{m}x^{m-2} & -x^2y^{\prime\prime} & = \sum \limits_{m=2}^\infty -m(m-1)a_m x^m \\
		-xy^\prime & = \sum \limits_{m=1}^\infty -a_m x^m & n^2y & = \sum \limits _{m=0}^\infty n^2 a_m x^m & 
	\end{align*}
	All of our powers of $x$ are the same except for the first term, which we change in the following manner 
	$$y^{\prime\prime} = \sum \limits _{m=2}^\infty m(m-1)a_{m}x^{m-2} = \sum \limits _{m=0}^\infty (m+2)(m+1)a_{m+2}x^{m}$$
	Since these the sum of these power series is identically zero, each term in the power series must be zero. From the constant term we know that
	\begin{align*}
		(2)(1)a_2 + n^2 a_0 & = 0 \\
		a_2 = \tfrac{-n^2}{2}a_0
	\end{align*}
	From the $x^1$ term we have
	\begin{align*}
		(3)(2)a_3 - a_1 + n^2a_1 & = 0 \\
		a_3 & = \tfrac{1-n^2}{6}a_1
	\end{align*}
	From $x^m$ where $m>1$ we have
	\begin{align*}
		(m+2)(m+1)a_{m+2} -m(m-1)a_m - ma_m+n^2 a_m & = 0 \\
		(m+2)(m+1) a_{m+2} & = m(m-1) a_m + m a_m - n^2 a_m \\
		(m+2)(m+1) a_{m+2} & = (m^2 - n^2) a_m \\
		a_{m+2} & = \tfrac{m^2 - n^2}{(m+2)(m+1)}a_m
	\end{align*}
	From this recurrence relation it is easy to derive the following
	\begin{align*}
		a_4 & = \frac{2^2 - n^2}{(4)(3)}a_2 \\
		& = \frac{2^2 - n^2}{(4)(3)}\frac{0^2-n^2}{(2)(1)}a_0 \\
		a_6 & = \frac{4^2-n^2}{(6)(5)} a_4 \\
		& = \frac{4^2-n^2}{(6)(5)}\frac{2^2 - n^2}{(4)(3)}\frac{0^2-n^2}{(2)(1)}a_0
	\end{align*}
	One can show using induction that 
	$$
	a_{2m} = \frac{\prod\limits_{i=0}^{m-1} \Big( (2i)^2-n^2 \Big)}{(2m)!}a_0
	$$
	Similarly
	\begin{align*}
		a_3 & = \frac{1^2 - n^2}{(3)(2)}a_1 \\
		a_5 & = \frac{3^2-n^2}{(5)(4)} a_3 \\
		& = \frac{3^2-n^2}{(5)(4)}\frac{1^2 - n^2}{(3)(2)}a_0
	\end{align*}
	Again one can show using induction that 
	$$
	a_{2m+1} = \frac{\prod\limits_{i=0}^{m-1} \Big( (2i+1)^2-n^2 \Big)}{(2m+1)!}a_1
	$$
	We now have a general solution to (\ref{p1eq2}) in terms of $a_0$ and $a_1$. To calculate a few terms when $n=3$, we have 
	\begin{align*}
		a_2 & = -\tfrac{3^2}{2}a_0 = -\tfrac{9}{2}a_0 \\
		a_3 & = \tfrac{1-3^2}{6}a_1 = -\tfrac{4}{3}a_1 \\
		a_4 & = \tfrac{2^2 - 3^2}{12}a_2 = \tfrac{45}{24}a_0 \\
		a_5 &= \tfrac{3^2 - 3^2}{20}a_3 = 0 \\
		a_6 & = \tfrac{4^2 - 3^2}{30}a_4 = \tfrac{315}{720}a_0 \\
	\end{align*}
	Thus 
	$$
	y = a_0 + a_1x -\tfrac{9}{2}a_0 x^2 -\tfrac{4}{3}a_1 x^3 + \tfrac{45}{24}a_0 x^4 + \tfrac{315}{720}a_0 x^6 + \dots
	$$
	
%*********************************************************************************************************************************************
\problem{1 d)} Prove the recurrence relationship 
	$$
	f_{n+1}(x) - 2xf_n(x) + f_{n-1}(x) = 0
	$$
	\begin{proof}
		Recall the angle sum formula $\cos(x+y) = \cos(x)\cos(y) - \sin(x)\sin(y)$. \\
		From this we calculate the following
		\begin{align*}
			f_{n+1}(x) & = \cos((n+1)\arccos(x)) \\
			& = \cos(n \arccos(x) + \arccos(x)) \\
			& = \cos(n\arccos(x))\cos(\arccos(x)) - \sin(n \arccos(x))\sin(\arccos(x))\\
			& = xf_n(x) - \sin(n \arccos(x))\sin(\arccos(x)) \\
			f_{n-1}(x) & = \cos((n-1)\arccos(x)) \\
			& = \cos(n \arccos(x) - \arccos(x)) \\
			& = \cos(n\arccos(x))\cos(-\arccos(x)) - \sin(n \arccos(x))\sin(-\arccos(x))\\
			& = xf_n(x) + \sin(n \arccos(x))\sin(\arccos(x))
		\end{align*}
		This last setp is because cosine is even, and sine is odd ($\cos(-x) = \cos(x)$ and $\sin(-x) = -\sin(x)$).
		Summing the two we have
		$$
		f_{n+1}(x) + f_{n-1}(x) = 2xf_n(x)
		$$
		Finally we have our result $f_{n+1}(x) - 2xf_n(x) + f_{n-1}(x) = 0$.
	\end{proof}

%*********************************************************************************************************************************************
\problem{1 e)} Prove the orthogonality of the functions (\ref{p1eq1}) with respect to the weighting function $(1-x^2)^{-\tfrac{1}{2}}$ on the interval $[-1,1]$.

	\begin{proof}
		Given $ n \neq m$, we would like to show that 
		$$
		\int_{-1}^{1} f_n(x)f_m(x)(1-x^2)^{-\tfrac{1}{2}} dx = 0
		$$
		Let $u = \arccos(x)$. Then $\tfrac{du}{dx} = (1-x^2)^{-\tfrac{1}{2}}$ and our equation becomes
		\begin{align*}
			\int_{-1}^{1} f_n(x)f_m(x)(1-x^2)^{-\tfrac{1}{2}} dx & = \int_{-1}^{1} \cos(nu)\cos(mu) \tfrac{du}{dx} dx \\
			& = \int_{-\pi}^{0} \cos(nu)\cos(mu) du \\
			& = \tfrac{1}{2}\int_{-\pi}^{0} \cos\Big((n-m)u \Big) + \cos\Big((n+m)u \Big) du \\
			& = \tfrac{1}{2} \Big[ \tfrac{1}{n-m}\sin\Big((n-m)u \Big) + \tfrac{1}{n+m}\sin\Big((n+m)u \Big) \Big]_{-\pi}^{0} \\
			& = 0
		\end{align*}
	\end{proof}
	
%*********************************************************************************************************************************************
\problem{1 f)} For $m=n$, compute the value of the integral
	$$
	\int_{-1}^{1} f_n(x)f_m(x)(1-x^2)^{-\tfrac{1}{2}} dx = 0
	$$
	As before we have 
	\begin{align*}
		\int_{-1}^{1} f_n(x)f_m(x)(1-x^2)^{-\tfrac{1}{2}} dx & = \int_{-\pi}^{0} \cos(nu)\cos(mu) du \\
		& = \int_{-\pi}^{0} \cos^2(nu)du \\
		& = \int_{-\pi}^{0} \tfrac{1}{2}(1+\cos(2nu))du \\
		& = \int_{-\pi}^{0} \tfrac{1}{2} + \tfrac{1}{2}\cos(2nu))du \\
		& = \Big[ \tfrac{1}{2}u + \tfrac{1}{4n}\sin(2nu) \Big]_{-\pi}^{0} \\
		& = \tfrac{\pi}{2}
	\end{align*}
	
%*********************************************************************************************************************************************
\problem{2)} Consider the differential equation
	\begin{align} \label{p3eq}
		2xy^{\prime\prime} + y^\prime + xy = 0
	\end{align}
	Find the general solution of (\ref{p3eq}) by seeking solutions in the form
	$$
	y(x) = x^r \sum_{m=0}^\infty a_m x^m
	$$
	If we assume that our solution is of the form above, then from the text we know that
	\begin{align*}
		y^\prime & = \sum_{m=0}^\infty(m+r)a_mx^{m+r-1} \\
		y^{\prime\prime} & = \sum_{m=0}^\infty(m+r)(m+r-1)a_mx^{m+r-2}
	\end{align*}
	Then the terms of our differential equation can be written as follows
%	\begin{align*}
%		x^2y^{\prime\prime} & = \sum_{m=0}^\infty(m+r)(m+r-1)a_mx^{m+r} \\
%		x\tfrac{1}{2}y^\prime & = 
%	\end{align*}
	\begin{align*}
		2xy^{\prime\prime} & = \sum_{m=0}^\infty 2(m+r)(m+r-1)a_mx^{m+r-1} \\
		y^\prime & = \sum_{m=0}^\infty(m+r)a_mx^{m+r-1} \\
		xy & = \sum_{m=0}^\infty a_m x^{m+r+1}
	\end{align*}
	We must align the expoenent on $x$ and the indicies of summation.
	\begin{align*}
		2xy^{\prime\prime} & = \sum_{m=0}^\infty 2(m+r)(m+r-1)a_mx^{m+r-1} \\
		& = 2(r)(r-1)a_0x^{r-1} + 2(1+r)ra_1x^{r} + \sum_{m=2}^\infty 2(m+r)(m+r-1)a_mx^{m+r-1} \\
		& = 2(r)(r-1)a_0x^{r-1} + 2(1+r)ra_1x^{r} + \sum_{m=0}^\infty 2(m+2+r)(m+2+r-1)a_{m+2}x^{m+2+r-1} \\
		& = 2(r)(r-1)a_0x^{r-1} + 2(1+r)ra_1x^{r} + \sum_{m=0}^\infty 2(m+r+2)(m+r+1)a_{m+2}x^{m+r+1} \\
		y^\prime & = \sum_{m=0}^\infty(m+r)a_mx^{m+r-1} \\
		& = ra_0x^{r-1} + (1+r)a_1x^{r} + \sum_{m=2}^\infty(m+r)a_mx^{m+r-1} \\
		& = ra_0x^{r-1} + (r+1)a_1x^{r} + \sum_{m=0}^\infty(m+r+2)a_{m+2}x^{m+r+1} 
	\end{align*}
	Since our power series is identically zero, we know each term is zero. From the $x^{r-1}$ term we have
	\begin{align*}
		2(r)(r-1)a_0 + ra_0 & = 0 \\
		(2r^2 - 2r + r)a_0 = 0 \\
		(2r^2 - r) a_0 = 0
	\end{align*}
	Since $r$ is chosen such that $a_0 \neq 0$ we know that $(2r^2 - r)=0$ and thus $r=0$ or $r=\tfrac{1}{2}$. This is case one from Theorem 2 in chapter 5.3 of the text, where our roots are distinct and do not differ by an integer. 
	
	Consider the case when $r=0$. Then our terms become
	\begin{align*}
		2xy^{\prime\prime} & = 2(r)(r-1)a_0x^{r-1} + 2(1+r)ra_1x^{r} + \sum_{m=0}^\infty 2(m+r+2)(m+r+1)a_{m+2}x^{m+r+1} \\
		& = \sum_{m=0}^\infty 2(m+2)(m+1)a_{m+2}x^{m+1} \\
		y^\prime & = ra_0x^{r-1} + (r+1)a_1x^{r} + \sum_{m=0}^\infty(m+r+2)a_{m+2}x^{m+r+1} \\
		& = a_1 + \sum_{m=0}^\infty(m+2)a_{m+2}x^{m+1} \\
		xy & = \sum_{m=0}^\infty a_m x^{m+1}
	\end{align*}
	Then we know that $a_1 = 0$ by looking at the constant terms. For the higher order terms we know that
	\begin{align*}
		2(m+2)(m+1)a_{m+2} + (m+2)a_{m+2} + a_m & = 0 \\
		(m+2)(2(m+1) + 1)a_{m+2} & = -a_m \\
		a_{m+2} & = \frac{-a_m}{(m+2)(2m+3)}
	\end{align*}
	Since $a_1=0$, $a_m = 0$ for all odd $m$. It can be easily seen with induction that for even $m$
	$$
	a_{2n} = \frac{a_0(-1)^{n}}{\prod\limits_{i=0}^{n-1}(2i+2)(4i+3)}
	$$
	Thus we know that one solution to (\ref{p3eq}) is $y_1 =  \sum_{m=0}^\infty a_m x^m$ where 
	\begin{align*}
		a_0 & = a_0 & a_1 & = 0 \\
		a_2 & = \tfrac{-1}{(2)(3)}a_0 = \tfrac{-1}{6}a_0 & a_3 & = 0 \\
		a_4 & = \tfrac{1}{(2)(3)(4)(7)}a_0 = \tfrac{1}{168}a_0 & a_5 & = 0 \\
		a_6 & = \tfrac{-1}{(2)(3)(4)(7)(6)(11)}a_0 = \tfrac{-1}{11,088}a_0 & a_7 & = 0 \\
		& \vdots & & \vdots
	\end{align*}
	To find a second solution we let $r = \tfrac{1}{2}$. Then our terms are
	\begin{align*}
		2xy^{\prime\prime} & = -\tfrac{1}{2}A_0x^{-\frac{1}{2}} + \tfrac{3}{2}A_1x^{\frac{1}{2}}\sum_{m=0}^\infty 2(m + \tfrac{5}{2}) (m + \tfrac{3}{2}) A_{m+2} x^{m+\frac{3}{2}} \\
		y^\prime & = \tfrac{1}{2}A_0x^{-\frac{1}{2}} + \tfrac{3}{2}A_1x^{\frac{1}{2}} + \sum_{m=0}^\infty(m+\tfrac{5}{2})A_{m+2}x^{m+\frac{3}{2}} \\
		xy & = \sum_{m=0}^\infty A_m x^{m+\frac{3}{2}}
	\end{align*}
	Collecting the $x^\frac{1}{2}$ terms we find that
	\begin{align*}
		\tfrac{3}{2}A_1 + \tfrac{3}{2}A_1 = 0 \\
		A_1 = 0
	\end{align*}
	Collecting the higher power terms we find that
	\begin{align*}
		2(m+\tfrac{5}{2})(m+\tfrac{3}{2})A_{m+2} + (m+\tfrac{5}{2})A_{m+2} + A_m & = 0 \\
		(m+\tfrac{5}{2})(2(m+\tfrac{3}{2})+1)A_{m+2} & = - A_m \\
		A_{m+2} & = \frac{-A_m}{(m+\tfrac{5}{2})(2m+4)} \\
		A_{m+2} & = \frac{-A_m}{\tfrac{1}{2}(2m+5)(2m+4)} \\
		A_{m+2} & = \frac{-2A_m}{(2m+5)(2m+4)} \\
	\end{align*}
	Again, since $A_1=0$, $A_m = 0$ for all odd $m$. It can be easily seen with induction that for even $m=2n$
	$$
	A_{2n} = A_0 \frac{(-2)^n}{\prod\limits_{i=0}^{n-1}(4i+5)(4i+4)}
	$$
	Thus we know that one solution to (\ref{p3eq}) is $y_2 =  x^{-\frac{1}{2}} \sum\limits_{m=0}^\infty A_m x^m$ where 
	\begin{align*}
		A_0 & = A_0 & A_1 & = 0 \\
		A_2 & = \tfrac{-2}{(5)(4)}A_0 = \tfrac{-1}{10}A_0 & A_3 & = 0 \\
		A_4 & = \tfrac{4}{(5)(4)(9)(8)}A_0 = \tfrac{1}{360}A_0 & A_5 & = 0 \\
		A_6 & = \tfrac{-8}{(5)(4)(9)(8)(13)(12)}A_0 = \tfrac{-1}{2,160}A_0 & A_7 & = 0 \\
		& \vdots & & \vdots
	\end{align*}
	
	Our general solution is $y = C_1 y_1(x) + C_2 y_2(x)$ where $C_1$ and $C_2$ are constants and $y_1$ and $y_2$ are given above.
	
%*********************************************************************************************************************************************
\problem{3 a)} Let $C$ be a counterclockwise-oriented simple closed curve in the $x$-$y$ plane and let $R$ be the interior of $C$. Use Green's Theorem to prove 
$$
\text{area of }R = \tfrac{1}{2}\int_C(xdy-ydx)
$$

	\begin{proof}
		Consider the vector field $F(x,y,z) = [M(x,y), N(x,y), 0]^T = [x, y, 0]^T$. Then
		\begin{align*}
			\tfrac{1}{2}\int_C(xdy-ydx) & = \tfrac{1}{2}\int_C(Mdy-Ndx) \\
			& = \tfrac{1}{2} \iint_R (\tfrac{\partial M}{\partial x} + \tfrac{\partial N}{\partial y}))dA \text{\space \space \space(by Green's Theorem)}\\
			& = \tfrac{1}{2} \iint_R (\tfrac{\partial x}{\partial x} + \tfrac{\partial y}{\partial y}))dA \\
			& = \tfrac{1}{2} \iint_R (1 + 1)dA \\
			& = \iint_R 1 dA = \text{area of }R
		\end{align*}
	\end{proof}

%*********************************************************************************************************************************************
\problem{3 b)} Use (a) to prove that the area of a circie is $A=\pi r^2$.

	\begin{proof}
		Consider a circle of radius $r$. Define a set of axes with origin at the center of the circle. Then $r(t) = [r\cos(t), r\sin(t)]^T$ for $-\pi \leq t \leq \pi$ is a parametrization of the circle. By part (a) we have
		\begin{align*}
			\text{area of }R & = \tfrac{1}{2}\int_C(xdy-ydx) \\
			& = \tfrac{1}{2}\int_{-\pi}^{\pi} r\cos(t)r\cos(t)-r\sin(t)(-r\sin(t))dt \\
			& = \tfrac{1}{2} \int_{-\pi}^{\pi} r^2dt \\
			& = \tfrac{1}{2} (\pi r^2 - (-\pi)r^2) \\
			& = \pi r^2
		\end{align*}
	\end{proof}
	
%*********************************************************************************************************************************************
\problem{3 c)} Use (a) to prove that the area of a rectangle is $A=ab$ where $a$ and $b$ are lengths of adjacent sides.

	\begin{proof}
		Difine a set of axies at one vertex of the rectangle such that the $x$ axis is along the side of length $a$ and the $y$ axis is along the side of length $b$. Define the curves $C_1, C_2, C_3$ and $C_4$ to be the sides of the rectangle counting counter clockwise around the rectangle starting at the origin. A parametirization of each curve along with the deriviatives would be
		\begin{align*}
			r_1(t) & = [at,0]^T & 0 & \leq t \leq 1 & r_1^\prime(t) & = [a,0]^T\\
			r_2(t) & = [a,bt]^T & 0 & \leq t \leq 1 & r_2^\prime(t) & = [0,b]^T\\
			r_3(t) & = [a-at,b]^T & 0 & \leq t \leq 1 & r_3^\prime(t) & = [-a,0]^T\\
			r_4(t) & = [0,b-bt]^T & 0 & \leq t \leq 1 & r_4^\prime(t) & = [0,-b]^T
		\end{align*}
		Then from part (a) we have
		\begin{align*}
			\text{area of }R & = \tfrac{1}{2}\int_C(xdy-ydx) \\
			& = \tfrac{1}{2}\int_{C_1}(xdy-ydx) +\tfrac{1}{2}\int_{C_2}(xdy-ydx) +\tfrac{1}{2}\int_{C_3}(xdy-ydx) \\ 
			& \text{\space\space\space\space} +\tfrac{1}{2}\int_{C_4}(xdy-ydx) \\
			& = \tfrac{1}{2}\int_{0}^1(at)(0)-(0)(a) dt + \tfrac{1}{2}\int_{0}^1(a)(b)-(bt)(0) dt + \\
			& \text{\space\space\space\space} \tfrac{1}{2}\int_{0}^1(a-at))(0)-(b)(-a) dt + \tfrac{1}{2}\int_{0}^1(0)(-b)-(b-bt)(0) dt \\
			& = \tfrac{1}{2}\int_{0}^1 2ab \text{ } dt \\
			& = ab
		\end{align*}
	\end{proof}

%*********************************************************************************************************************************************
\problem{3 d)} Use (a) to prove that the area of a triangle is $A=\tfrac{bh}{2}$ where $h$ is the height and $b$ is the length of the base.

	\begin{proof}
		Create a set of axes such that the $x$ axis runs along the base of the triangle and the origin is at one of the verticies such that all of the points of the triangle are in the first quadrant. Then the coordinates of the verticies are $(0,0)$, $(b,0)$, and $(k,h)$ for some positive $k$. Define the curves $C_1, C_2$ and $C_3$ to be the sides of the triangle counting counter clockwise around the triangle starting at the origin. A parametirization of each curve along with the deriviatives would be
		\begin{align*}
			r_1(t) & = [bt,0]^T & 0 & \leq t \leq 1 & r_1^\prime(t) & = [b,0]^T\\
			r_2(t) & = [b-(b-k)t,ht]^T & 0 & \leq t \leq 1 & r_2^\prime(t) & = [k-b,h]^T\\
			r_3(t) & = [k-kt,h-ht]^T & 0 & \leq t \leq 1 & r_3^\prime(t) & = [-k,-h]^T\\
		\end{align*}
		Then from part (a) we have
		\begin{align*}
			\text{area of }R & = \tfrac{1}{2}\int_C(xdy-ydx) \\
			& = \tfrac{1}{2}\int_{C_1}(xdy-ydx) +\tfrac{1}{2}\int_{C_2}(xdy-ydx) +\tfrac{1}{2}\int_{C_3}(xdy-ydx) \\ 
			& = \tfrac{1}{2}\int_{0}^1(bt)(0)-(0)(b) dt + \tfrac{1}{2}\int_{0}^1(b-(b-k)t)(h)-(ht)(k-b) dt \\
			& \text{\space\space\space\space} + \tfrac{1}{2}\int_{0}^1(k-kt)(-h)-(h-ht)(-k) dt \\
			& = \tfrac{1}{2}\int_{0}^1 bh - bht + kht - kht + bht -hk + hkt +hk - hkt \text{ } dt \\
			& = \tfrac{1}{2}\int_{0}^1 bh \text{ } dt \\
			& = \frac{bh}{2}
		\end{align*}
	\end{proof}
	
%*********************************************************************************************************************************************
\hspace{-4 ex} \large \textbf{Honesty Pledge:}
	I have neither given nor received unauthorized aid on this exam.\\
	\begin{tabular}{c}
		\text{\space\space\space\space\space\space\space\space\space\space\space\space\space\space\space\space\space\space\space\space\space\space\space\space\space\space\space\space\space\space\space\space\space\space\space\space\space}\\
		\hline
	\end{tabular}\\
	Sage Shaw

\end{document}
