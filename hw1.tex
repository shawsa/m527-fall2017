\documentclass{article}

\usepackage{fancyhdr} 
%\usepackage{mathtools}
\usepackage{amsmath}

\pagestyle{fancy}
\fancyhf{}
\rhead{Shaw \space \thepage}

\setlength\parindent{0pt}

\begin{document}
	\thispagestyle{empty}
	
	\begin{flushright}
		Sage Shaw \\
		m527 - Fall 2017 \\
		Aug. 30, 2017
	\end{flushright}
	
	{\large \textbf{HW1: Section 5.1: 2, 4, \& 10}}\bigbreak
	\textbf{2-4} \space Determine the radius of convergence.\bigbreak
	\textbf{2.} $\sum\limits_{m=0}^{\infty} (m+1)mx^{m}$
	
	
	After simplifying $\sum\limits_{m=0}^{\infty} (m+1)mx^{m} = \sum\limits_{m=0}^{\infty} (m^{2}+m)x^{m}$, we apply the ratio test as follows:
	\begin{equation*}
	\begin{split}
	\lim_{m\to\infty} \left\lvert \frac{((m+1)^{2}+(m+1))x^{n+1}}{(m^{2}+m)x^{n}}\right\rvert
	& = \lvert x \rvert * \lim_{m\to\infty} \frac{m^{2}+3m+2}{m^{2}+m} \\
	& = \lvert x \rvert * \lim_{m\to\infty} \frac{m^{2}+3m+2}{m^{2}+m} * \frac{m^{-2}}{m^{-2}} \\
	& = \lvert x \rvert * \lim_{m\to\infty} \frac{1+3m^{-1}+2m^{-2}}{1+m^{-1}} \\
	& = \lvert x \rvert\\
	\end{split}
	\end{equation*}
	
	The ratio test concludes that the series converges when $\lvert x \rvert < 1$. Thus $\sum\limits_{m=0}^{\infty} (m+1)mx^{m}$ is convergent for $ -1 < x < 1$ and the radius of and the radius of convergence is 1.
	
	\bigbreak

	
	\textbf{4.} $\sum\limits_{m=0}^{\infty} \frac{x^{2m+1}}{(2m+1)!}$
	
	Apply the ratio test as follows:
	\begin{equation*}
	\begin{split}
	\lim_{m\to\infty} \left\lvert 
	\frac{
		\frac{x^{2(m+1)+1}}{(2(m+1)+1)!}
	}{
		\frac{x^{2m+1}}{(2m+1)!}
	}
	\right\rvert & = 
	\lim_{m\to\infty} \left\lvert \frac{(2m+1)!x^{2m+3}}{(2m+3)!x^{2m+1}} \right\rvert \\
	& = \lvert x^{2} \rvert \lim_{m\to\infty}  * \frac{(2m+1)!}{(2m+3)!} \\
	& = \lvert x^{2} \rvert  \lim_{m\to\infty} * \frac{(2m+1)!}{(2m+3)(2m+2)(2m+1)!} \\
	& = \lvert x^{2} \rvert  \lim_{m\to\infty} * \frac{1}{(2m+3)(2m+2)} \\
	& = 0
	\end{split}
	\end{equation*}
    
    The ratio test concludes that the series converges when $0 < 1$. Thus $\sum\limits_{m=0}^{\infty} \frac{x^{2m+1}}{(2m+1)!}$ is convergent for all $x$ and the radius of convergence is infinite. 
    
    \bigbreak
    
    \textbf{10.} Find a power series solution in powers of $x$ for $y''-y'+xy=0$.
    
    Assume $y=\sum\limits_{n=0}^{\infty}a_{n}x^{n} = a_{0} + a_{1}x + \sum\limits_{n=2}^{\infty}a_{n}x^{n}$.
    
    Then $y'= a_{1} + \sum\limits_{n=2}^{\infty}na_{n}x^{n-1}$ and $y''= \sum\limits_{n=2}^{\infty}n(n-1)a_{n}x^{n-2}$.
    
    From these we can derive the following set of equations:
    \begin{equation*}
    \begin{split}
    	y''& = 0 + 0 + \sum\limits_{n=2}^{\infty}n(n-1)a_{n}x^{n-2} \\
    	-y' & = 0 + -a_{1} + \sum\limits_{n=2}^{\infty}-na_{n}x^{n-1} \\
    	xy & = a_{0}x + a_{1}x^{2} + \sum\limits_{n=2}^{\infty}a_{n}x^{n+1} \\
    \end{split}
    \end{equation*}
    
    From our premise, we have 
    \begin{equation}
    \begin{split}
    	0 & = y''-y'+xy \\
    	& = (\sum\limits_{n=2}^{\infty}n(n-1)a_{n}x^{n-2}) +(-a_{1} + \sum\limits_{n=2}^{\infty}-na_{n}x^{n-1}) + (a_{0}x + a_{1}x^{2} + \sum\limits_{n=2}^{\infty}a_{n}x^{n+1}) \\
    	& = -a_{n} + a_{0}x + a_{1}x^{2} + \sum\limits_{n=2}^{\infty}n(n-1)a_{n}x^{n-2} + \sum\limits_{n=2}^{\infty}-na_{n}x^{n-1} + \sum\limits_{n=2}^{\infty}a_{n}x^{n+1}
    \end{split}
    \end{equation}
    
    We would like each summation term to have identical powers of x at each index and start at the same index. Let us label each term as follows 
    \begin{equation}
    \begin{split}
    	A & = \sum\limits_{n=2}^{\infty}n(n-1)a_{n}x^{n-2} \\
    	B & = \sum\limits_{n=2}^{\infty}-na_{n}x^{n-1} \\
    	C & = \sum\limits_{n=2}^{\infty}a_{n}x^{n+1}
    \end{split}
    \end{equation}
    
    From equations $(1)$ and $(2)$ we have 
    \begin{equation}
    	0 = -a_{1} + a_{0}x + a_{1}x^{2} + A + B + C.
    \end{equation}
    
    By choosing the index substitution $k=n-2$, we obtain the following representation for $A$:
    \begin{equation}
    \begin{split}
		A & = \sum\limits_{n=2}^{\infty}n(n-1)a_{n}x^{n-2} \\
		& = \sum\limits_{k=0}^{\infty}(k+2)(k+1)a_{k+2}x^{k} \\
		& = \sum\limits_{k=0}^{\infty}(k^{2}+3k+2)a_{k+2}x^{k} \\
		& = 2a_{2} + 6a_{3}x + 12a_{4}x^{2} + \sum\limits_{k=3}^{\infty}(k^{2}+3k+2)a_{k+2}x^{k}
	\end{split}
    \end{equation}
    
    By choosing the index substitution $k=n-1$, we obtain the following representation for $B$:
    \begin{equation}
    \begin{split}
		B & = \sum\limits_{n=2}^{\infty}-na_{n}x^{n-1} \\
		& = \sum\limits_{k=1}^{\infty}-(k+1)a_{k+1}x^{k} \\
		& = -2a_{2}x - 3a_{3}x^{2} + \sum\limits_{k=3}^{\infty}-(k+1)a_{k+1}x^{k}
	\end{split}
    \end{equation}
    
    By choosing the index substitution $k=n+1$, we obtain the following representation for $C$:
    \begin{equation}
    \begin{split}
		C & = \sum\limits_{n=2}^{\infty}a_{n}x^{n+1} \\
		& = \sum\limits_{k=3}^{\infty}a_{k-1}x^{k}
	\end{split}
    \end{equation}
    
    Substituting equations $(4)$, $(5)$, and $(6)$, into $(3)$ we obtain
    \begin{equation}
    \begin{split}
    	0 = & -a_{1} + a_{0}x + a_{1}x^{2} + A + B + C. \\
    	= & -a_{1} + a_{0}x + a_{1}x^{2} + (2a_{2} + 6a_{3}x + 12a_{4}x^{2} + \sum\limits_{k=3}^{\infty}(k^{2}+3k+2)a_{k+2}x^{k}) \\
    	& \space \space + (-2a_{2}x - 3a_{3}x^{2} + \sum\limits_{k=3}^{\infty}-(k+1)a_{k+1}x^{k}) + (\sum\limits_{k=3}^{\infty}a_{k-1}x^{k}) \\
    	= & (2a_{2}-a_{1}) + (a_{0}+6a_{3}-2a_{2})x  + (a_{1}+12a_{4}-3a_{3})x^{2} \\
    	& + \sum\limits_{k=3}^{\infty}[(k^{2}+3k+2)a_{k+2} -(k+1)a_{k+1} + a_{k-1}]x^{k}
    \end{split}
    \end{equation}
    
    Since the power series in equation $(7)$ is identically zero, each of the coefficients of the power series is zero:
    \begin{equation}
    \begin{split}
    	0 = & 2a_{2}-a_{1} \\ 
    	0 = & a_{0}+6a_{3}-2a_{2} \\
    	0 = & a_{1}+12a_{4}-3a_{3} \\
    	0 = & (k^{2}+3k+2)a_{k+2} -(k+1)a_{k+1} + a_{k-1} \quad \textrm{for}k  \quad \geq 3
    \end{split}
    \end{equation}
    
    Solving in terms of $a_{0}$ and $a_{1}$ we obtain
    \begin{equation}
    \begin{split}
    	a_{2} & = \frac{a_{1}}{2} \\ 
    	a_{3} & = \frac{a_{1}-a_{0}}{6} \\
    	a_{4} & = \frac{-a_{1}-a_{0}}{24} \\
    	a_{k} & = \frac{-a_{k-3}+(k-1)a_{k-1}}{k^{2}+3k+2} \quad \textrm{for} \quad k \geq 5
    \end{split}
    \end{equation}
    These equations uniquely determines a power series solution in terms of two constants $a_{0}$ and $a_{1}$ as is consistent for a general solution to a second order partial differential equation.
\end{document}
